\documentclass{beamer}

% Theme choice:
\usetheme{CambridgeUS}
\usepackage{amsmath,amssymb}
% Title page details: 
\title{Assignment 4} 
\author{Nishanth Bhoomi (CS21BTECH11040)}
\date{\today}
\logo{\large \LaTeX{}}


\begin{document}

\begin{frame}
    \titlepage 
\end{frame}

\logo{}

\let\vec\textbf

\begin{frame}{Outline}
    \tableofcontents
\end{frame}



\section{Problem }
\begin{frame}{Problem }

    \begin{block}{}
      Using the central limit theorem. show that for large n:
        \begin{equation}
            \frac{c^n}{(n-1)!}x^{n-1}e^{-cx} \approx \frac{c}{\sqrt{2\pi n}}e^{-(cx-n)^2/2n}
        \end{equation}
    \end{block}
\end{frame}

\begin{frame}{Solution}
    \begin{alertblock}{Erlang Density Function}
        For $N=0,1,2,3,\ldots$  and constant $a>0$
        \begin{align}
            f_X(x) &= \frac{a^N}{(N-1)!}a^{N-1}e^{-ax}\\
            \bar{X} &=\frac{N}{a} \hspace*{17pt} \sigma^2 =\frac{N}{a^2}\\
            \Phi_X(\omega) &= \left(\frac{a}{a-j\omega}\right)^N
        \end{align}
    \end{alertblock}

\end{frame}
\begin{frame}{Solution}
    \begin{alertblock}{Characteristics function and Normality}
        The characteristic function of a random vector is by definition the function:
        \begin{equation}
            \Phi(\Omega) = E(e^{j\Omega \vec{X}'}) = E(e^{j(\omega_1x_1+\omega_2x_2+\ldots+\omega_n x_n)}) = \Phi(j\Omega)
        \end{equation}
        where,
        \begin{equation}
            \vec{X} = [x_1,\dots,x_n] \hspace*{12pt} \Omega = [\omega_1,\ldots,\omega_n]
        \end{equation}
        Let $z=x_1+x_2+\ldots+x_n$ be a random variable\\
        Now, If the random variables $x_i$ are independent with respective densities $f_i(x_i)$, then
        \begin{equation}
            E(e^{j(\omega_1x_1+\omega_2x_2+\ldots+\omega_n x_n)}) = E(e^{j\omega_1x_1})\ldots E(e^{j\omega_nx_n})
        \end{equation}
    \end{alertblock}
\end{frame}
    \begin{frame}{Solution}
        \begin{alertblock}{}
        Hence,
        \begin{equation}
            \Phi_z(\omega) = E(e^{j(\omega_1x_1+\omega_2x_2+\ldots+\omega_n x_n)}) = \Phi_1(\omega)\ldots \Phi_n(\omega)
        \end{equation}
        where $\Phi_i(\omega)$ is the chacteristic function of $x_i$\\
        On Applying the convolution theorem for Fourier transform, we obtain
        \begin{equation}
            f_z(z) = f_1(z)*\ldots*f_n(z)
        \end{equation}
    \end{alertblock}
    \begin{alertblock}{Central Limit Theorem}
        Given $n$ independent random variable $x_i$, we form their sum
        \begin{equation}
            x=x_1+\ldots+x_n
        \end{equation}
        This is a random variable with mean $\eta = \eta_1+\ldots+\eta_n$ and variance $\sigma^2=\sigma_1^2+\ldots+\sigma_n^2$
    \end{alertblock}
\end{frame}
\begin{frame}{Solution}
    \begin{alertblock}{}
        The Central Limit Theorem (CLT) states that under certain general conditions, the distribution $F(x)$ of $x$ approaches a normal distribution with the same mean and variance as $n$ increases.
        \begin{equation}
            F(x) \simeq G\left(\frac{x-\eta}{\sigma}\right)
        \end{equation}
        Futhermore, if the random variables $x_i$ are of continous type, the density $f(x)$ of $x$ approaches a normal density.
        \begin{equation}
            f(x) \simeq \frac{1}{\sigma\sqrt{2\pi}}e^{-(x-\eta)^2/2\sigma^2}
        \end{equation}
    \end{alertblock}
\end{frame}
\begin{frame}{Soltuion}
    \begin{block}{}
        Let $x_1,\ldots,x_n$ be some i.i.d random variables with exponential density function $\lambda e^{-\lambda x}$ and characteristic function $\Phi_{x_i}(\omega)$. Consider a random variable $X = x_1+\ldots+x_n$.
        \begin{equation}
            \Phi_{x_i}(\omega) = \frac{a}{a-j\omega}
        \end{equation}
        Since from equation (9), we have,
        \begin{equation}
            \Phi_X(\omega) = \Phi_{x_1}(\omega)\ldots \Phi_{x_n}(\omega) = \left(\frac{a}{a-j\omega}\right)^n
        \end{equation}
        On applying Convolution Theorem, we obtain
        \begin{equation}
            f_X(x) = \frac{a^n}{(n-1)!}a^{n-1}e^{-ax} \hspace*{12pt} \textbf{(Erlang Density Function)}
        \end{equation}
    \end{block}
\end{frame}
\begin{frame}{Solution}
    \begin{block}{}
        On Applying Central Limit Theorem on random variable $X$ (For some Large Value of $n$). we get
        \begin{equation}
            f_X(x) = \frac{1}{\sqrt{2\pi \sigma^2}}e^{-(x-\eta)^2/2\sigma^2}
        \end{equation}
        So On Comparing equation (16) and (17), we conclude:\\
        Erlang Density Function approaches Normal Density Curve with same mean and same variance at large value of $n$.
        \begin{equation}
           \frac{c^n}{(n-1)!}c^{n-1}e^{-cx} \approx \frac{1}{\sqrt{2\pi \frac{n}{c^2}}}e^{(x-\frac{n}{c})^2/2\frac{n}{c^2}} =\frac{c}{\sqrt{2\pi n}}e^{-(cx-n)^2/2n} 
        \end{equation}
        Hence Proved.
    \end{block}
\end{frame}
\end{document}

