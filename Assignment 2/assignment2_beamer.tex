\documentclass{beamer}
\usetheme{CambridgeUS}

\setbeamertemplate{caption}[numbered]{}

\usepackage{enumitem}
\usepackage{amsmath}
\usepackage{amssymb}
\usepackage{gensymb}
\usepackage{graphicx}
\usepackage{txfonts}

\def\inputGnumericTable{}

\usepackage[latin1]{inputenc}                                 
\usepackage{color}                                            
\usepackage{array}                                            
\usepackage{longtable}                                        
\usepackage{calc}                                             
\usepackage{multirow}                                         
\usepackage{hhline}                                           
\usepackage{ifthen}
\usepackage{caption}

\title{AI1110 \\ Assignment 2}
\author{Nishanth Bhoomi \\ CS21BTECH11040}
\date{}
\begin{document}
	% The title page
	\begin{frame}
		\titlepage
	\end{frame}
	
	% The table of contents
	\begin{frame}{Outline}
    		\tableofcontents
	\end{frame}
	
	% The question
	\section{Question}
	\begin{frame}{Question 12}

A cone is inscribed in a sphere of radius 12 cm.If the volume of the cone is maximum,find its height.\\
    \end{frame}
    
    % The solution
	\section{Solution}
	\begin{frame}{Solution}
	
\begin{align}
\end{align}
\begin{figure}[ht]
    \centering
    \includegraphics[width=2in]{fig assign2.png}
	\caption{Cone inscribed in a sphere}
	\label{Fig-1}
 \end{figure}
\end{frame}

\begin{frame}{}
Let radius of cone be r and its height be h.\\
Here, r^2 + x^2 =12^2\\
      r^2 = 144 - x^2\\
      
$Now, the  volume  of the cone = \frac{1}{3} \times \pi \times r^2 \times h \\
                           $ = \frac{1}{3} \times \pi \times (144-x^2) \times (12+x)$\\ 
                            = $\frac{1}{3} \times \pi \times (12-x) \times (12+x)^2$\\



Now,$\frac{dV}{dx}=\frac{d}{dx}(\frac{1}{3}\times\pi\times(12-x)\times(12+x)^2)$ \\
$= \frac{1}{3} \times \pi \times(-(12+x)^2 + 2\times(144-x^2))$\\

$\frac{d^2V}{dx^2}=\frac{1}{3} \times \pi \times(-2\times(12+x)+2\times(0-2x))$\\

                 =$\frac{1}{3}  \times \pi \times(-24-2x-4x)$\\
                 =$\frac{1}{3} \times \pi \times-(24+6x)$\\
                 
\end{frame}{}

\begin{frame}
                 
Here,second derivative is negative,implies volume of cone is maximum.
Now, put $\frac{dV}{dx} =0$ , we obtain,

$\frac{\pi}{3} \times{-(12+x)^2+2\times(144-x^2)}$\\
\implies (12+x) \times{(-(12+x) + 2\times(12-x)} =0\\
\implies (12+x) \times{-12-x+24-2\times x}=0\\
\implies (12+x) \times (12-3x) =0\\
\implies$(12+x =0)     or    (12-3x =0)$\\
\implies $x = -12     or    x = 4$\\

\end{frame}
% The final answer
	\section{Answer}
	\begin{frame}{Answer}
	

	As distance cannot be negative, x = 4.\\
Hence, for maximum volume of cone, the height of the cone is h=12+x where x=4cm\\
So, height of the cone is 16cm.\\


\end{frame}
\end{document}