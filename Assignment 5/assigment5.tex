%%%%%%%%%%%%%%%%%%%%%%%%%%%%%%%%%%%%%%%%%%%%%%%%%%%%%%%%%%%%%%%
%	
% Welcome to Overleaf --- just edit your LaTeX on the left,
% and we'll compile it for you on the right. If you open the
% 'Share' menu, you can invite other users to edit at the same
% time. See www.overleaf.com/learn for more info. Enjoy!
%
%%%%%%%%%%%%%%%%%%%%%%%%%%%%%%%%%%%%%%%%%%%%%%%%%%%%%%%%%%%%%%%

% Inbuilt themes in beamer
\documentclass{beamer}

%packages:
% \usepackage{tfrupee}
% \usepackage{amsmath}
% \usepackage{amssymb}
% \usepackage{gensymb}
% \usepackage{txfonts}

% \def\inputGnumericTable{}

% \usepackage[latin1]{inputenc}                                 
% \usepackage{color}                                            
% \usepackage{array}                                            
% \usepackage{longtable}                                        
% \usepackage{calc}                                             
% \usepackage{multirow}                                         
% \usepackage{hhline}                                           
% \usepackage{ifthen}
% \usepackage{caption} 
% \captionsetup[table]{skip=3pt}  
% \providecommand{\pr}[1]{\ensuremath{\Pr\left(#1\right)}}
% \providecommand{\cbrak}[1]{\ensuremath{\left\{#1\right\}}}
% %\renewcommand{\thefigure}{\arabic{table}}
% \renewcommand{\thetable}{\arabic{table}}      

\setbeamertemplate{caption}[numbered]{}

\usepackage{enumitem}
\usepackage{tfrupee}
\usepackage{amsmath}
\usepackage{amssymb}
\usepackage{gensymb}
\usepackage{graphicx}
\usepackage{txfonts}

\def\inputGnumericTable{}

\usepackage[latin1]{inputenc}                                 
\usepackage{color}    
\usepackage{textcomp, gensymb}         
\usepackage{array}                                            
\usepackage{longtable}                                        
\usepackage{calc}                                             
\usepackage{multirow}                                         
\usepackage{hhline}                             
\usepackage{mathtools}
\usepackage{ifthen}
\usepackage{caption} 
\providecommand{\pr}[1]{\ensuremath{\Pr\left(#1\right)}}
\providecommand{\cbrak}[1]{\ensuremath{\left\{#1\right\}}}
\renewcommand{\thefigure}{\arabic{table}}
\renewcommand{\thetable}{\arabic{table}}   
\providecommand{\brak}[1]{\ensuremath{\left(#1\right)}}

% Theme choice:
\usetheme{CambridgeUS}

% Title page details: 
\title{AI1110 \\ Assignment-5} 
\author{Nishanth Bhoomi CS21BTECH11040}
\date{\today}
\logo{\large \LaTeX{}}


\begin{document}

% Title page frame
\begin{frame}
    \titlepage 
\end{frame}
\logo{}


% Outline frame
\begin{frame}{Outline}
    \tableofcontents
\end{frame}



\section{Question}
\begin{frame}{Question}
    \begin{block}{} 
    The process $s(t)$ is shot noise with $\lambda =3$ where $h(t)=2$ for $0 \le t \le 10$ and $ h(t)=0$ otherwise. Find $ E\{s(t)\},E\{s^2(t)\},P\{s(7)=0\}$ 
                  \end{block}
     
\end{frame}



\section{Solution}
\begin{frame}{Solution}
\begin{block}{}
Given $\lambda=3$ ,\\
$h(t)=2     (0\le t\le 10),$\\
$h(t)=0$ otherwise.\\
\end{block}
\begin{block}{}
\eta _s=E\{s(t)\}=\lambda\(\int_{0}^{10} h(t) \,dt\)= \lambda\(\int_{0}^{10} 2 \,dt\)=3\times 2(10-0)=3\times20=60

\sigma_s^2=var\{s(t)\}=\lambda\(\int_{0}^{10} h^2(t) \,dt\)=\lambda\(\int_{0}^{10} 4 \,dt\)=3\times4(10-0)=120


$E\{s^2(t)\} - E\{s(t)\}^2=var\{s(t)\}\\
E\{s^2(t)\} &= E\{s(t)\}^2+var\{s(t)\}\\
E\{s^2(t)\} &= 3600+120\\
E\{s^2(t)\}=3720$
\end{block}
\end{frame}
\begin{frame}{}
\begin{block}{}
 Finding  P\{s(7)=0\}:
\end{block}
       As s(7)=0 if there are no points in the interval(7-10,7).the number of points in this interval is a poisson RV with parameter 10\lambda=30.\\
        
    Hence,\\ P\{s(7)=0\}=e^-^3^0
        
        \end{frame}
        
     
\end{document}
