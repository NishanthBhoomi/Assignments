\documentclass{beamer}

% Theme choice:
\usetheme{CambridgeUS}
\usepackage{amsmath}

\usepackage{array}
\newcolumntype{P}[1]{>{\centering\arraybackslash}p{#1}}

% Title page details: 
\title{Probability and Random Variables}
\subtitle{Assignment 3}
\author{Nishanth Bhoomi (cs21btech11040)}
\date{\today}


\begin{document}

% Title page frame
\begin{frame}
    \titlepage 
\end{frame}

% Outline frame
\begin{frame}{Outline}
    \tableofcontents
\end{frame}

\section{Problem}

\begin{frame}{Assignment 3}
  \frametitle{Problem}
 In the coin-tossing experiment, the probability of heads equals p and the probability of tails equals q. We define the random variable x such that\\
x(h)=1, x(t)=0\\
We shall find it's distribution function F(x) for every x' from $-\infty$ to $\infty$.\\

 \end{frame}
  
\section{Solution}
\begin{frame}

\frametitle{Solution}

\begin{figure}[h]
    \includegraphics[width=0.47\textwidth]{As3.1.png}
     \includegraphics[width=0.47\textwidth]{As3.2.png}
\end{figure}

If x $\geq$ 1, then x(h)=1 $\seq$ x and x(t)=0 $\leq$ x.\\
Hence F(x)=P\{x $\leq$ x\} = P(h,t)=1 , x$\geq$1 \\
If 0 $\leq$ x $\leq$ 1, then x(h)=1 and x(t)=0 $\leq$x. Hence\\
F(x)=P\{x $\leq$ x\} = P\{t\}=q , 0$\leq$ x $\leq$1.\\
If x$<$0, then x(h)=1 $>$ x and x(t)=0 $>$ x. Hence,
F(x)=P\{x$\leq$x\}=P\theta\}= 0, x<0
\end{frame}

\end{document}
  
